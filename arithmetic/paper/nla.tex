\section{Non-Linear Arithmetic}
\label{sec:nla}

Similar to solving for integer feasibility, the arithmetic solver
solves constraints over polynomials using a waterfall model for non-linear
constraints.
At the basis it maintains, for every monomial 
term $x \cdot x \cdot y$, a definition
$m = x \cdot x \cdot y$, where $m$ is a variable
that represents the monomial $x \cdot x \cdot y$.
The module for non-linear arithmetic then collects the monomial
definitions that are violated by the current evaluation,
that is $\eval{m} \neq \eval{x} \cdot \eval{x} \cdot \eval{y}$. It  attempts to 
establish a valuation $\beta'$ where $\beta'(m) = \beta'(x) \cdot \beta'(x) \cdot \beta'(y)$, 
or derive a consequence that no such evaluation exists.



\subsection{Patch Monomials}

A \emph{patch} for a variable $x$ is \emph{admissible} if the update $\eval{x} := v$
does not break any integer linear constraints and $x$ does not occur in monomial
equations that are not already false under $\beta$.

\begin{itemize}
\item Set $\eval{m} := \eval{x} \cdot \eval{x} \cdot \eval{y}$ and check if the patch of $m$ is admissible.
\item Try to set $\eval{y} := \eval{m} / (\eval{x} \cdot \eval{x})$, provided $\beta{x}$ is not 0,
  and check that the patch for $x$ is admissible.
\item When $\eval{m} = r^2$ for a rational and $m := x \cdot x$ try patching $x$ by setting $\eval{x} := \pm r$.
\end{itemize}


\subsection{Bounds propagation}
A relatively inexpensive step is to propagate and check bounds based 
on non-linear constraints. For example, for $y \geq 3$, then $m = x\cdot x\cdot y \geq 3$,
if furthermore $x \leq -2$, we have the strengthened bound $m \geq 12$.
Bounds propagation can also flow from bounds on $m$ to bounds on the 
variables that make up the monomial, such that when $m \geq 8, 1 \leq y \leq 2, x \leq 0$, 
then we learn the stronger bound $x \leq -2$ on $x$. It uses an interval arithmetic abstraction,
that understands bounds propagation over squares. Thus, if $-2 \leq x \leq 2$, then $0 \leq x^2 \leq 4$
instead of $-4 \leq x^2 \leq 4$.

The solver also performs Horner expansions of polynomials to derive stronger bounds.
For example, if $x \geq 2, y \geq -1, z \geq 2$, then $y + z \geq 1$ 
and therefore $x\cdot (y + z) \geq 2$, but we would not be
able to deduce this fact if combining bounds individually for $x\cdot y$ 
and $x \cdot z$ because no bounds can be inferred for $x \cdot y$ in isolation.
The solver therefore attempts different re-distribution of multiplication
in an effort to find stronger bounds.


\subsection{Adding bounds}
Non-linear bounds propagation only triggers if all variables are either bounded from above or below or occur with an even power.
The solver includes a pass where it adds a bound case split $x \geq 0$ to variables $x$ where $lo_x = -\infty, hi_x = +\infty$.
The added case split may help trigger bounds propagation, such as detecting conflicts on $xy > 0, xz > 0, y > 0 > z$.


\subsection{Gr{\"o}bner reduction}

Z3 uses a best effort Gr\"obner basis reduction to find inconsistencies, cheaply, 
and propagate consequences. While Gr{\"o}bner basis heuristics are not new to Z3,
they have evolved and to our knowledge the integration is unique among SMT solvers.
Recall that reduced Gr{\"o}bner basis for a set of polynomial equations $p_1 = 0, \ldots, p_k = 0$
is a set $q_1 = 0, \ldots, q_m = 0$, such that every $p_i$ is a linear sum of $q_j$'s,
and the leading monomials of every pair $q_i, q_j$, $i \neq j$, have no common factors.
Since Z3 uses completion as a heuristic to make partial inferences, it does not seek to compute
a basis.
The Gr\"obner module performs a set of partial completion steps, preferring
to eliminate variables that can be isolated, and expanding a bounded number of super-position
steps (reductions by S-polynomials).

Z3 first adds equations $m = x_1\ldots x_k$ for monomial definitions that are violated.
It then traverses the transitive cone of influence of Simplex rows that contain one of the added variables
from monomial definitions. It only considers rows where the basic variable is bounded.
Rows where the basic variable is unbounded are skipped
because the basic variable can be solved for over the reals.
Fixed variables are replaced by constants, and the bounds constraints that fixes the variables
are recorded as dependencies with the added equation. Thus, the equations handled by the
Gr{\"o}bner basis reduction are of the form $\langle p_i: xy + 3z + 3 = 0, d_i: \{3 \leq u \leq 3\} \rangle$, where $p_i$
is a polynomial and $d_i$ is a set of dependencies corresponding to fixed variables that were replaced
by constants in $p_i$. In the example, we replaced $u$ by $3$ and the definition $\langle m = xy, \emptyset\rangle$ resolved $m$ by $xy$.
Dependencies are accumulated when two polynomials are resolved to infer a new derived equality.
Generally, when $\langle xy + p_1 = 0, d_1\rangle, \langle xz + p_2 = 0, d_2 \rangle$ are two polynomial equations,
then $\langle zp_1 - yp_2 = 0, d_1 \cup d_2\rangle$ can be derived accumulating the premises $d_1, d_2$.

Finally, equations are pre-solved if they are linear and can be split
into two groups, one containing a single variable that has a
lower (upper) bound, the other with more than two variables
with upper (lower) bounds. This avoids losing bounds information
during completion.

After (partial) completion, the derived equations are post-processed:
\begin{description}

\item[Constant propagation]
For equalities of the form $x = 0$ or $ax + b = 0$. If the current assignment to $x$ does not
satisfy the equation, then the equality is propagated as a lemma.

\item[Linear propagation]
As a generalization of constant propagation, if the completion contains linear equations that
evaluate to false under the current assignment, then these linear equations are added to the
Simplex Tableau. Example~\ref{ex:yoav} illustrates a use where this propagation is useful.

\item[Factorization] Identify factors of the
form $x y p \simeq 0$ where $x, y$ are variables an $p$ is linear. We infer 
the clause $x y p \simeq 0 \implies x \simeq 0 \lor y \simeq 0 \lor p \simeq 0$.

\end{description}

\begin{example}[Combining Gr{\"o}bner completion and Linear Solving]
\label{ex:yoav}
We include an example obtained from Yoav Rodeh at Certora.
The instance was not solvable prior to adding simplex propagation.
To solve it, Certora relied on treating multiplication as an uninterpreted function and
including selected axioms for modular arithmetic and multiplication that were instantiated by E-matching.
The distilled example is:

\[
   L \leq x \cdot y \leq U \land 1 \leq x \land m_r \leq U \land x \cdot y \neq m_r
\]
where $L = N \idiv 2, U = 1 + L$, $m_r = (x \cdot (\mathit{ite}(y \geq 0, y, N + y))) \mod N$.
We assume $N$ is even, such as $N = 2^{256}$.
The solver associates a variable $m$ with $x \cdot y$ and $m'$ with $x \cdot y'$ and $y'$ with $\mathit{ite}(y \geq 0, y, N + y)$
and includes the constraints $0 \leq m_r < N, m_q \cdot N + m_r = m'$, where $m_q$ is an integer variable.
The most interesting case is where $y < 0$, so $y' = y + N$. Gr{\"o}bner basis completion
then allows to derive $m_q N + m_r = m' = x(y + N) = xy + xN = m + xN$, which by integer linear arithmetic reasoning (the extended GCD test)
contradicts $m \neq m_r$ because the absolute value of both variables is below $N$.
\end{example}

Our extraction of linear constraints represents a partial integration of linear programming and polynomial arithmetic,
that favors only including linear inequalities over variables and monomials that are already present. Our implementation
does not include any variables for new monomials produced by completion.
In comparison, the approach in~\cite{DBLP:journals/pacmpl/KincaidKZ23} proposes a domain for abstract interpretation that populates a linear
solver with all equations produced by a completion. We have not experimented in depth with extending our approach with a full basis,
or use it as a starting point for finding lemmas based on Positivstellensatz or other extension mechanisms~\cite{DBLP:conf/csl/Tiwari05,DBLP:conf/cade/PlatzerQR09}.





\usetikzlibrary{shapes,arrows}
\usetikzlibrary{positioning}

\tikzstyle{block} = [rectangle, draw, text centered, rounded corners, minimum height=2em]
\tikzstyle{line} = [draw, -latex']

\begin{wrapfigure}{r}{0.4\textwidth} 
  \begin{center}
    \begin{tikzpicture}[node distance = 3em, scale = 0.2]
  \node [circle, draw] (xroot) {$x$};
  \node [circle, below of = xroot, left of= xroot, draw] (xnext) {$x$};
  \node [circle, below of = xnext, left of= xnext, draw] (y1) {$y$};
  \node [circle, below of = xnext, right of = xnext] (yempty) {};
  \node [circle, right of= yempty, draw] (y2) {$y$};
  \node [circle, below of = y1, left of = y1] (t1) {$5$};
  \node [circle, below of = y1, right of = y1] (t2) {$0$};
  \node [circle, right of = t2] (t3) {$1$};
  \node [circle, below of = y2, right of = y2] (t4) {$1$};
  \path [line] (xroot.west) -- (xnext.north);
  \path [line] (xroot.east) -- (y2.north);
  \path [line] (xnext.west) -- (y1.north);
  \path [line] (xnext.east) -- (y2.north);
  \path [line] (y1.west) -- (t1.north);
  \path [line] (y1.east) -- (t2.north);
  \path [line] (y2.west) -- (t3.north);
  \path [line] (y2.east) -- (t4.north);

\end{tikzpicture}

  \end{center}
  \vspace{1pt}
  \caption{PDD representation of $5x^2y + xy + y + x + 1$\label{fig:pdd}}
\end{wrapfigure}
We use an adaptation of ZDD (Zero suppressed decision diagrams~\cite{Minato93,NishinoYMN16}) to represent polynomials.
The representation has the advantage that polynomials are stored in a shared data-structure and operations
over polynomials are memoized. A polynomial over the real is represented as an acyclic graph, where 
nodes are labeled by variables and edges are labeled by coefficients. Figure~\ref{fig:pdd} shows a polynomial stored
in a polynomial decision diagram, PDD.


The root node labeled by $x$ represents the polynomial $x\cdot l + r$, 
where $l$ is the polynomial of the left sub-graph and $r$ the polynomial
of the right sub-graph. The left sub-graph is allowed to be labeled again by $x$, 
but the right sub-graph may only have nodes labeled by variables that are smaller
in a fixed ordering. The fixed ordering used in this example sets $x$ above $y$.
Then the polynomial for the right sub-graph is $y + 1$, and the polynomial with the
left sub-graph is $5xy + (y + 1)$.




\subsection{Incremental linearization}
Following~\cite{CimattiGIRS18} we incrementally linearize monomial definitions that
currently evaluate to false.
For example, we include lemmas of the form $x = 0 \rightarrow m = 0$
and $x = 1 \rightarrow m = y$, for $m = x^2y$.
Incremental linearization proceeds by first applying linearizations that are considered cheap,
such as case splitting on whether variables take values 0, 1, -1, when these boundary conditions
are exhausted, instantiates lemmas based on monotonicity of multiplication and tangents.
It is possible that there are overlapping monomial definitions, such as $m' = x \cdot y$.
Then incremental linearization takes into account that the definition for $m$ can be
\emph{factored} into $m' \cdot x$. It also uses specialized congruence closure reasoning,
recognizing equalities modulo signs, such that when $m = x \cdot y, m' = z \cdot y$ and
$x = -z$ in the current context, then $m \sim -m'$.

% LN, not sure if it is a good place, and 
To find all factorizations of monomial $m = \prod_{i \in A}{x_i} $ as $m = m_0 \cdot m_1$, we choose $a \in A$ and enumerate over all proper subsets $B$ of $A$ containing $a$. For each $B$ we check that $m_0 = \prod_{i \in B}{x_i}$ and $m_1 = \prod_{i \in A \setminus B}{x_i}$ are monomials.

To support floating point arithmetic reasoning
we also include incremental linearization lemmas for special
cases of exponentiation~\cite{DBLP:journals/tocl/CimattiGIRS18}. 
We also added rules for incremental linearization of divisibility operations. 
The front-end to the core arithmetic solver axiomatizes integer and real division operations using multiplication and addition,
so that the solver does not have to reason about division. Nevertheless, we found use cases for
instantiating axioms of the form $y > 0 \land x > z \implies x/y > z/y$ (when the input contains terms $x/y, z/y$)
bypassing indirect reasoning around constraints created by axioms.



\subsection{NLSat}
As an end-game attempt, the solver attempts to solver the non-linear constraints using a complete solver
for Tarski's fragment supported by the NLSat solver~\cite{JovanovicM12}. NLSAT is complete for non-linear arithmetic
and includes branch-and-bound to handle cases of integer arithmetic. It can therefore sometimes be used to solve
goals, bypassing the partial heuristics entirely. The solver therefore includes selected calls to NLSat with a small
resource bound to close branches before attempting incomplete heuristics such as incomplete linearization.
The results in Section~\ref{sec:eval} suggests that our use of NLSat with a resource bound currently incurs
significant overhead on easy problems, but overall is an advantage. We found that it is sometimes the case that turning off NLSat
all-together can speed up the solver significantly, but is overall a disadvantage.

