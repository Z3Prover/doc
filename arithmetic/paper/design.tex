\section{Design Goals and Implementation Choices}

The SMT formalism for arithmetic in many cases subsumes formalisms used by mixed integer, MIP, solvers.
However, there are several fundamental differences between the workloads we have tuned the arithmetic
solver for compared to workloads seen by MIP solvers.
Z3 uses infinite precision ``big-num'' numeral representations, in contrast to using floating points.
The drawback is that the arithmetic solver is impractical on linear programming optimization problems,
but the engine avoids having to compensate for rounding errors and numerical instability.
The solver uses a sparse matrix representation
for the Dual Simplex tableau.
We also created a version that uses an LRU decomposition and floating
point numbers but found that extending this version with efficient backtracking was not practical
compared to the straight-forward sparse matrix format.
Finally, the solver remains integrated within a CDCL engine that favors an eager case split strategy
leaving it to conflict analysis to block infeasible branches. This contrasts mainstream MIP designs
that favor a search tree of relatively few branches where the engine performs significant analysis before case splits.

%The arithmetic solver reduces operations such as integer modulus, remainder, and division into equations.
%It eagerly compiles bounds axioms of the form $x \geq 3 \implies x \geq 2$ for atomic formulas $x \geq 3, x\geq 2$.

