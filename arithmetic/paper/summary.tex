\section{Summary and Discussion}

We presented the architecture and a cross-cut of system innovations in a new arithmetic solver in Z3.
It is shown to provide good advances relative to the legacy arithmetic solver,
and our evaluation suggests it compares very well with other state-of-art SMT solvers.
The new solver enabled us to addresses some design choices with the previous solver that limited extensibility.
Notably, the new solver separates its representation of arithmetic constraints from terms shared by other solvers through an E-graph.
We noticed that limitation of using shared terms is that the boundary for when to treat a sub-term as a variable or a polynomial is inherently
ambiguous. The legacy solver is also highly incomplete for non-linear
reasoning (over the reals). 

Many avenues for further innovations and tuning remain. Another important aspect is trust.
Implementing the many features of the arithmetic solver is inherently a complex task.
Many bugs get uncovered by fuzzing~\cite{DBLP:conf/sat/BrummayerLB10,DBLP:conf/sigsoft/MansurCWZ20,winterer-zhang-su-oopsla2020,winterer-zhang-su-pldi2020,DBLP:journals/pacmpl/ParkWZS21,DHuang,Maolin23} both in the legacy and new solver, 
bearing witness to the difficulty of creating a correct solver. The solver therefore supports a number of ways to validate results.
The easiest validation is for satisfiable formulas, where the satisfiable formula is model checked against the returned model.
The main difficulty with satisfiable models is to correctly track interpretations of under-specified operations, such as division by 0.
To check that consequences produced by the solver are valid, there is a self-validator 
enabled by the {\tt smt.arith.validate=true}. It uses the legacy arithmetic solver
to check lemmas and propagations. 
There is also a mechanism for creating certificates that can be processed offline or online.
With each theory axiom and propagation produced by the solver, it produces a certificate object that can be used to validate
inferences by the arithmetic solver. The certificates are exposed in proof objects~\cite{MouraB08b} and also as annotations in proof logs~\cite{z3prooflogs}.
Z3 contains a built-in proof checker for proof logs. The proof checker for arithmetic certificates validates
conflicts that can be justified by using Farkas lemma and bounds propagations that use cuts. It currently falls back to invoking
Z3 on lemmas (using the legacy arithmetic solver) for non-linear lemmas and other cases not covered by the built-in checker.
Certificates created for {\tt QF\_LRA} are fully handled, while self-contained or independent proof checking for more expressive fragments
of arithmetic is future work.
