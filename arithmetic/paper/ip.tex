\section{Integer Linear Arithmetic}
\label{sec:ip}

The mixed integer linear solver consists of several layers that first attempt to \emph{patch} integer variables
form solutions over reals to solutions over integers. Then, if patching fails to correct all integer variables,
it checks for integer infeasibility by checking light-weight Diophantine feasibility criteria and then resort to variants of
Gomory Cuts and Branch and Bound.


\subsection{Patching}


In a feasible tableau we can assume that all non-basic variables are at their bounds
and therefore if they have integer sort they are assigned integer values.
Only the basic variables could be assigned non-integer values.
Patching seeks changing values of non-basic values in order to assign integer values to basic variables.
A related method, that diversifies values of variables using \emph{freedom} intervals was
described in~\cite{MouraB08}, but we found it does not preserve integral assignments.

Thus, we patch rows with basic variables $x_b \not\in\Int$.
We use a process that seeks a $\delta$, such that $|\delta|$ is minimal
and the row with $x_b$ is of the form $x_b + \alpha y + \alpha'x' = 0$,
where $\alpha \not\in \Int$,
such that the update $\eval{y} := \eval{y} + \delta$ is within the bounds of $y$,
$x_b$ is assigned an integer value and such that $x_b$ becomes integer without 
breaking any bounds in the tableau.
We describe details of the patching method in Appendix~\ref{app:ip-patching}.


\begin{example}
  Suppose we are given a tableau of the form
  $
 y - \frac{1}{2} x = 0, \ 
 z - \frac{1}{3} x  = 0
$
where $y, z$ are basic variables and $x$ has bounds $[3,10]$, $y$ has bounds $[-3,4]$, $z$ has bounds $[-5,12]$.
The variable $x$ is initially assigned at the bound $\eval{x} = 3$. Then $\eval{y} = \frac{3}{2}$ and
$\eval{z} = 1$. But neither $y$ nor $z$ is close to their bounds. We can move $x$ to $8$
without violating the bound for $y$ because of $y - \frac{1}{2} x = 0$.
Thus, the freedom interval for $x$ is the range $[3,8]$ and within this range there is a solution,
$x = 6$, where $y, z$ are integers and within their bounds.
\end{example}



\subsection{Cubes}
An important factor in solving more satisfiable integer arithmetic instances is
a method by Bromberger and Weidenbach~\cite{BrombergerW16,BrombergerW17}.
It allows detecting feasible inequalities over integer variables 
by solving a stronger linear system.
Their method relies on the following property: The inequalities $Ax \leq b$
are integer feasible, for matrix $A$ and vectors $x, b$, if
$Ax \leq b - \frac{1}{2}\onenorm{A}$ has a solution over the reals.
We use the 1-norm $\onenorm{A}$ of a matrix
as a column vector, such that each entry $i$ is the sum of the absolute
values of the elements in the corresponding row $A_i$.

\begin{example}
Suppose we have $3x + y \leq 9 \wedge - 3y \leq -2$ and wish to find an integer solution. 
By solving $3x + y \leq 9 - \frac{1}{2}(3 + 1) = 7, -3y \leq -2 - \frac{1}{2}3 = -3.5$ we find
a model where $y = \frac{7}{6}, x = 0$. After rounding $y$ to $1$ and maintaining $x$ at $0$ we obtain an
integer solution to the original inequalities.
\end{example}

Z3 includes a twist relative to~\cite{BrombergerW17} that allows to avoid strengthening on selected inequalities~\cite{Vampire17Theorem}.
First, we note that \emph{difference} inequalities of the form $x - y \leq k$, where $x, y$ are integer variables
and $k$ is an integer offset need not be strengthened: they have a solution over reals if and only if they have a solution over integers.
For octagon constraints $\pm x \pm y \leq k$,
there is a boundary condition: they need only require strengthening if $x, y$ are assigned at mid-points
between integral solutions. For example, if $\eval{x} = \frac{1}{2}$ and $\eval{y} = \frac{3}{2}$, for $x + y \leq 2$.


\subsection{GCD consistency}\label{sec:gcd-consistency}
A basic test for integer infeasibility is by enforcing divisibility constraints.
\begin{example}
  Assume we are given a row $5/6x + 3/6y + z + 5/6u = 0$, where $x, y$ are fixed at $2 \leq x \leq 2$, $-1 \leq u \leq -1$, and $z$ is the base variable.
  Then it follows that $5 + 3(y + 2z) = 0$ which has no solution over the integers:
  The greatest common divisor of coefficients to the non-fixed variables ($3$) does not divide the constant offset from the fixed variables ($5$).
\end{example}
The basic test is extended as follows. For each row $a x + b y + c = 0$, where
\begin{itemize}
  \item $a, b, c$ and $x, y$ are vectors of integer constants and variables, respectively.
  \item the coefficients in $a$ are all the same and smaller than the coefficients in $b$
  \item the variables $x$ are bounded
\end{itemize}
Let $l := a\cdot lb(x), u := a \cdot ub(x)$.
That is, the lower and upper bounds for $a\cdot x$ based on the bounds for $x$.
If $\lfloor \frac{u}{\gcd(b,c)} \rfloor > \lceil  \frac{l}{\gcd(b,c)} \rceil$, then
there is no solution for $x$ within the bounds for $x$.



\subsection{Branching}
Similar to traditional MIP branch-and-bound methods, 
the solver creates somewhat eagerly case splits on bounds 
of integer variables if the dual simplex solver fails to assign them integer values. For example, Simplex may assign an integer variable $x$, the value $\frac{1}{2}$, in which case z3 creates a literal $x \leq 0$ that triggers two branches
$x \leq 0$ and $\neg(x \leq 0) \equiv x \geq 1$.


\subsection{Cuts}\label{sec:cuts}
The arithmetic solver produces Gomory cuts from rows where the basic variables are non-integers after
the non-basic variables have been pushed to the bounds. Z3 implements Chv{\'a}tal-Goromy cuts described in~\cite{DutertreM06}.
It also implements algorithms from~\cite{DilligDA09,ChristH15}
to generate cuts after the linear systems have been transformed into Hermitian matrices.
It is a long-standing and timely challenge~\cite{DBLP:conf/nips/BalcanPSV22} to harness the effectiveness of selecting cuts.
While the solver takes~\cite{DutertreM06} as starting point, it incorporates a few heuristics and enhancements.

Recall that a row $\sum_{j=0}^{k} a_j\cdot x_j + x_b = 0$ 
from the tableau is called a Gomory row, and is eligible for Gomory cut, if $x_b$ is a basic variable and $x_j$ are non-basic variables,
$x_b$ is an integral variable, but $\eval{x_b}$ is not integral, and for each $x_j$ we have $\eval{x_j} = lo_j$ or $\eval{x_j} = hi_j$,
and the bounds are not strict. 

We use a relaxed definition of Gomory rows. For a non-basic integral variable $x_j$ we allow for value $\eval{x_j}$ to be not at a bound when $\eval{x_j}$, and $a_j$ are both integers: Let us call such $x_j$ \emph{row integral}. We provide further justification for
this relaxed definition in Appendix~\ref{app:row-integral}.


To select a cut variable, our main heuristic sorts all Gomory rows from the tableau by the distance of $\eval{x_b}$ from the nearest integer, that is $\min \{ {\eval{x_b}-\lfloor \eval{x_b} \rfloor, \lceil \eval{x_b} \rceil - \eval{x_b}} \}$, and pick a few of them having the minimal distance to produce the cuts. We break the ties by preferring the variables that are used in more terms.
Heuristics used previously relied on distances to bounds.

We also look for the case when $x_b$ is at an extremum. For example, if for all $x_j$ we have $\eval{x_j} = lo_j$,
and $a_j > 0$ then $x_b$ is at the maximum, and we deduce $x_b \leq \lfloor \eval{x_b} \rfloor$.
The explanations of the Gomory term do not include constraints on $x_j$ for $j \in A$ from the relaxed definition,
but in case of an extremum these constraints should be added.

% NSB: does not make sense: if $\eval{x_b} = -\eval{\sum_{j=0}^{k} a_j\cdot x_j}$, then $\eval{x_b}$ will also have to be integral.
%Another consideration is the case of an empty Gomory term:
%it happens when $\sum_{j=0}^{k} a_j\cdot x_j$ is always integral. We report a conflict in this situation.

Cuts are consequences of the current bounds. By default the solver adds new rows to the Dual Simplex tableau corresponding to cuts,
but makes an exception when the new rows include large numerals.
In analogy the solver avoids bounds propagation, Section~\ref{sec:bounds-propagation},
when computation of bounds relies on big-num arithmetic.
Similarly, cuts that involve large coefficients are first added to a temporary scope where the tableau is checked for feasibility.
The cuts are only re-added within the main scope if the temporary tableau is infeasible.

