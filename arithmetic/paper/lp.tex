\section{Linear Real Arithmetic}
\label{sec:lp}
The solver first determines whether arithmetic constraints are feasibility over the reals.
It also attempts to propagate equalities eagerly for shared variables and infer stronger bounds of variables.

\subsection{Linear Solving}

Based on~\cite{DutertreM06} the solver for real linear inequalities uses a dual simplex solver.
It partitions the variables into \emph{basic} and \emph{non-basic} variables, and maintains
a global set of equalities of the form $x_{bi} + \sum_j a_{ij}x_j = 0$,
where $i$ refers to the $i$'th row, $x_{bi}$ is basic and $x_j$ range over non-basic variables.
It also maintains an evaluation
$\beta$, such that $\eval{x_{bi}} + \sum_j a_{ij}\eval{x_j}= 0$ for each row $i$.
Each variable $x_j$ is assigned lower and upper bounds during search.
The solver then checks whether $lo_j \leq \eval{x_j} \leq hi_j$, for bounds
$lo_j, hi_j$ that are dynamically added and removed by Boolean decisions $x_j \leq hi_j$, $x_j \geq lo_j$.
If the bounds are violated, it updates the evaluation and pivots if necessary.
We recall the approach using an example.

\begin{example}

For the following formula
\[
 y \geq 0 \land (x + y \leq 2 \lor x + 2y \geq 6) \land (x + y \geq 2 \lor x + 2y > 4)
\]
the solver introduces auxiliary variables $s_1, s_2$ and represents the formula as
\[
  x + y - s_1 = 0, x + 2y - s_2 = 0, \
  x \geq 0, (s_1 \leq 2 \vee s_2 \geq 6), (s_1 \geq 2 \vee s_2 > 4)
\]

Only bounds (e.g., $s_1 \leq 2$) are asserted during search.
The slack variables $s_1, s_2$ are initially basic (dependent) and $x, y$ are non-basic.
In dual Simplex tableaux, values of a non-basic variable
$x_j$ can be chosen between $lo_j$ and $hi_j$.
The value of a basic variable is a function of non-basic variable values.
Pivoting swaps basic and non-basic variables and moves basic variables within their bounds to bounds violations.
For example, assume we start with a set of initial values
$x = y = s_1 = s_2 = 0$
and bounds $x \geq 0, s_1 \leq 2, s_1 \geq 2$. 
Then $s_1$ has to be 2 and it is made non-basic. 
Instead, $y$ becomes basic:
$
{y} + x - s_1 = 0, \ {s_2} + x - 2s_1 = 0.
$
The new tableau updates the assignment of variables to
$x = 0, s_1 = 2, s_2 = 4, y = 2$. The resulting assignment
is a model for the original formula.

\end{example}

\subsection{Finding equal variables - cheaply}
It is useful to have the arithmetic solver propagate implied equalities when arithmetic is used in combination with other theories, or even when
it solves non-linear arithmetic constraints. Equality propagation is disabled for pure arithmetic theories, such as {\tt QF\_LIA, QF\_LRA}~\cite{SMTLIB2}.
Z3 originally used a method based on storing \emph{offset} equalities in a hash table.
An offset equality is of the form $x_i = y + k$, where $k$ is a numeric constant.
Offset equalities are extracted from rows that contain $x_i$ as a basic variable, and contains
only one other non-fixed variable $y$, while other variables are fixed and their lower
(upper) bounds add up to $k$. It turns out that computing $k$ is
expensive when the tableau contains large numeric constants. Hash table operations
contribute with additional overhead. It turns out that neither the offset hash-table, nor computing $k$,
is really necessary. We describe our new method, using an example.
We first described the method for avoiding to compute offsets in~\cite{DBLP:conf/sbmf/BjornerN20}.
The description there relies on building a tree data-structure for connecting variables and fails to leverage that the dual simplex tableau can be
used directly.


\begin{example}

From equalities $x + 1 = y, y - 1 = z$ infer that $x = z$. Based on the tableau form, the solver is presented with the original equality atoms via slack variables
\[
    s_1 = x + u - y, s_2 = y - u - z, 1 \leq u \leq 1
    \]	
The tableau can be solved by setting $x = 0, y = 0, z = 0, s_1 = 1, s_2 = -1, u = 1$.
The slack variables are bounded when the equalities are asserted
\[
    s_1 = x + u - y, s_2 = y - u - z, 1 \leq u \leq 1, 0 \leq s_1 \leq 0, 0 \leq s_2 \leq 0
    \]
The original solution is no longer valid, the values for $s_1, s_2$ are out of bounds.
Pivoting re-establishes feasibility using a different solution, for example
\[
    x = y - u - s_1, y = z - u - s_2, 1 \leq u \leq 1, 0 \leq s_1 \leq 0, 0 \leq s_2 \leq 0
    \]
with assignment $z = 0, x = y = -1$. The variables $x$ and $y$ have the same value,
but must they be equal under all assignments? We can establish this directly by subtracting the
right-hand sides $z - u - s_1$ and $z - u - s_2$ from another and by factoring in the constant bounds to obtain
the result $0$. But subtraction is generally expensive if there are many bounded constants in the rows.
Such arithmetical operations are not required to infer that $x = y$.

\end{example}

Z3 uses the following conditions to infer an equality between variables $x, y$ having the same values in the current assignment:
\begin{itemize}
    \item $x$ is basic, and the tableau has row $x - y + \alpha = 0$,
    \item $x, y$ are connected through a non-basic variable $z$ in a pair of the tableau rows in one of the following forms
	(1) $x - z + \alpha = 0, y - z + \alpha' = 0$, (2) $x + z + \alpha = 0, y + z + \alpha' = 0$, 	
\end{itemize}
where $\alpha, \alpha'$ are linear combinations of fixed variables.

We experimented with generalizing the connection between equal variables to
allow non-unit coefficients on $z$, but it did not result in measurable improvements.


\subsection{Bounds Propagation}
\label{sec:bounds-propagation}
It is not uncommon that SMT formulas contain different bounds for the same variable, such as one atom $x \geq 2$ and another atom $x \geq 3$.
When the atom $x \geq 3$ is assigned to true, the solver can directly propagate $x \geq 3$. Bounds can also be inferred indirectly.
With a row $x - 2y = 0$ and bound $y \geq 1$, it follows that $x \geq 2$. To implement direct bounds propagation, the solver
maintains an index that maps each variable to the set of bounds atoms where it occurs. To implement indirect bounds propagation,
the solver queries updated rows for whether they imply bounds that are stronger than the currently asserted bounds. If so, these
stronger bounds are used by the index for direct bounds propagation.
