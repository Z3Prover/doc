\section{Evaluation}
\label{sec:eval}

To get an idea of how the new solver compares and how the individual features of weigh on performance
we conducted a set of measurements. They are based on three benchmark sets: {\tt QF\_LIA}, SMTLIB2 benchmarks for the theory of quantifier-free integer linear arithmetic,
{\tt QF\_NIA}, SMTLIB2 benchmarks for the theory of quantifier-free non-linear integer arithmetic, and {\tt benchmark-submission}, a smaller set of verification
conditions obtained from Certora.
Data associated with the measurements summarized in this Section are available from~\cite{z3data}.
The measurements are listed in Appendix~\ref{app:eval}.
We ran the solvers for 600s and measured how many problems are solved within 600s.
We compared default settings of the solvers
with CVC5~\cite{DBLP:conf/tacas/BarbosaBBKLMMMN22,DBLP:conf/cade/KremerRBT22,cvc5tool},
MathSat5~\cite{DBLP:conf/tacas/CimattiGSS13,mathsattool}, and Yices2~\cite{DBLP:conf/cav/Dutertre14,yicestool}, and Z3's legacy arithmetic solver,
which is available by setting the option {\tt smt.arith.solver=2}. The advances relative to the legacy solver are noticable. Compared to other solvers,
Yices2 and MathSat5 shine as fast out of the gates solving relatively more problems within 1s,
but are mainly limited by the set of features it supports, such as lack of support for algebraic data-types.

The feature-wise evaluation suggests that using NLSat to eagerly close branches comes with a steep cost for easy benchmarks.
It can likely be tuned in future versions of Z3. The eager use of NLSat still provides an overall benefit. Z3 also uses \emph{tactics} that run
a few strategies with a 5 second resource bounds early on to find models using SAT encodings and selected branch-and-bound strategies. They are also
a cause of relatively slow startup. The default tactics can be overridden. The overall biggest impact feature is incremental
linearization. While it is run after gcd tests, bounds propagation and Gr{\"o}bner saturation, it has a significant effect. Other features have each a
relative minor effect in isolation. The solver relies on their cummulative effect.


